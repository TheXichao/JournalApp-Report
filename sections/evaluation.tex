% evaluation.tex
\chapter{Evaluation}

\section{Requirements Specification Evaluation}
From my testing section, it can be confirmed that all the originally proposed requirements from the specification have been fully met.

\section{Independent End-User Feedback}
A very critical friend of mine has tested and provided constructive feedback on the application. Here are the feedbacks:

\subsubitem{Positive Feedback}
\begin{itemize}
    \item all the features work as expected of a journal app
    \item the clean design of the website is a positive aspect
    \item liked the fact that the website was fast
    \item He thinks it's easy to use.
    \item the user liked the fact that the website was secure and the passwords are hashed
    \item liked how the entries are displayed
    \item thought the statistics page was a nice touch
\end{itemize}

\subsubitem{Constructive Criticism}
\begin{itemize}
    \item The error message after an API call is too "codey" \\
    If there is an error after an API call, I conditionally render the Error message by simply parsing it as a string and displaying it. A potential improvement could be creating error pages that display the error message based on the error code.

    \item The content box for the journal entry is too small \\
    In the future I could make a better UI for the journal entry page, potentially making my custom reusable input box and button components.

    \item The user would have liked to see individual journal entries better
    \item Editing journal entries
    \item Deleting journal entries \\
    At the moment the user can only view the journal entries and five entries are displayed on a page. The user would like to see individual journal entries better, edit, and delete them. This could be implemented by adding a button to each journal entry that allows the user to expand on the entry. Then the user can edit or delete the entry. I already have the API endpoints for deleting journal entries, but due to time constraints, I was not able to implement the Frontend for it.


    \item The user would have liked to go back to the home page after an invalid URL \\
    He tried to access an invalid URL, and the page was blank. I could implement a 404 page that redirects the user back to the home page.

    \item The user would like to search by date or search by journal \\
    A search bar could be implemented that allows the user to search by date or search by journal. This would be a useful design
\end{itemize}



\section{Improvements}
There are many potential improvements that could be made to the project. Here are some of them:

\begin{itemize}
    \item Mobile App \\
    The Backend is already very sophisticated with many endpoints, so a mobile app could be developed that uses the same API endpoints. This would allow users to write journal entries on the go. A mobile app could also have features like push notifications to remind the user to write a journal entry. In addition, a mobile app could allow users to write journals offline and sync when they are online.

    \item Email \\
    A feature that I have started implementing is sending emails to users daily to prompt them to write a journal entry. Then if they respond to the email, the email will be saved as a journal entry. This feature could be created via a cron job that runs every day and sends an email to all users who have not written a journal entry in the last 24 hours. This feature could be implemented using the Django Email API. 

    \item Sentiment Analysis \\
    I have tried making calls to the API provided by Google Cloud Natural Language API, but I haven't fully implemented it. This would be useful information to have providing insight into the user's overall mental health.

    \item Better Analysis \\
    The analysis section could be improved by adding more detailed statistics about the user's journal entries. For example, I could analyse the user's sentiment over time to let the user know if they are feeling better or worse. I could also analyse the user's most common words to see what they are writing about the most. 

    \item Photo and Audio Upload \\
    It would be cool to be able to include pictures or audio. Recently I found some old pictures and it made me feel nostalgic, so having the ability to include pictures or audio in a journal entry would be a nice feature to have. This could be implemented by allowing the user to upload a picture or audio file and then storing the URL in the database. Then the user could view the picture or listen to the audio in the journal entry.
    
\end{itemize}